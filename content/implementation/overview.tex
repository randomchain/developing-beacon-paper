\subsection{Overview}%
\label{sub:overview}
An overview of our implementation can be seen in \vref{fig:impl_figure}. This chapter will elaborate on the figure and explain choices.
\mtjnote{Some text giving an overview coming from the design to the implementation.}

% Not focus on outlets and usability applications (e.g. tracking your input from start to finish)
In the implementation of the beacon, we choose not to focus on usability applications, such as allowing a user to track their input automatically through the beacon.
Instead, we implement a beacon with simple, secure, and succinct operation.
It collects inputs, processes these inputs, does some computation, and outputs data through a set of outlets.
These steps will be elaborated in the following sections.

% All data out of the system will be correlated by sequence numbers (output, commit, proofs)
The data, which the beacon outputs each cycle consists of a few parts:
\begin{eletterate*}
\item a commitment, containing all used inputs in the computation;
\item an output, computed from the commitment; and
\item a set of proofs, which can be used to verify the computation.
\end{eletterate*}
This data is correlated with sequence numbers in order to keep track of matching parts of the output.

\subsection{Frameworks, Libraries, and Language Choice}
% SOA is achieved with concurrent and async message passing
To achieve the \acrfull{soa} and pipelining presented in~\cref{sec:design}, we utilize a framework for asynchronous message passing and concurrency.
This will allow us to develop the components separately and to gradually implement business logic by mocking not-yet-complete services, as long as the inter-component communications protocol and message passing method is agreed upon.
We choose to use \textit{ZeroMQ}\footnote{\url{http://zeromq.org/}} as this framework for message passing and concurrency.

\subsubsection{ZeroMQ}
% ZeroMQ because language agnostic and fast. Ready for real world beacon
\textit{ZeroMQ} does not get in the way of the implementation, meaning that it is language agnostic with bindings for virtually all programming languages, and that it is fast, flexible, and scalable.
The name \textit{ZeroMQ} hints at its alternative approach to a messaging framework, in that no (or \textit{zero}) broker is needed between components.
This means that our beacon implementation does not rely on any centralized broker for passing around messages --- no single point of failure in that aspect.

% Fan-in, fan-out pipeline. Good patterns from ØMQ
The pipelining with fan-in/fan-out at input processor and computation respectively \mtjnote{Fix wording such that figure is still relevant}, is implemented using the primitives of \textit{ZeroMQ}.
These primitives provide useful patterns for communication, e.g.\ pipeline and publish/subscribe patterns.
Using something like \textit{ZeroMQ} and not developing our own ad-hoc solution, allows us to leverage well-tested technologies, and focus on the beacon itself instead.

\subsubsection{Python}
% Python for components for fast prototyping. Other languages where performance is important
The components of our beacon are mainly implemented using Python\footnote{Python 3, of course, not Legacy-Python}; both for fast prototype turnaround, but also because of a state of the art \textit{ZeroMQ} library.
However, in some cases the performance overhead in Python is unsuitable for the task at hand.
Fortunately, Python programs can easily be extended with C code, and entire components can be implemented in this fashion if deemed necessary.
