\subsection{Infrastructure}%
\label{sub:infrastructure}
Establishing a solid and scalable infrastructure is a significant part of implementing our beacon.
We choose not to rely on running all components of the beacon on the same machine.

Because of this, we use TCP sockets through \textit{ZeroMQ}.
This means that operations such as reading bytes from the sockets and reconnecting in case of networking issues is handled by \textit{ZeroMQ}, which also maintains a local queue per socket to mitigate network congestion.
Sockets in \textit{ZeroMQ} require one end to bind and the other to connect, and usually it is recommended that the most stable or consistent component binds, while dynamic or unstable components connect.
Only one component can bind to a given socket while many can connect to that same socket.

In the \enquote{publish/subscribe} pattern \textit{ZeroMQ} handles the message routing based on subscription prefixes, which means less traffic on our network.
Furthermore, the fan-in pipelining is implemented with a \enquote{push/pull} socket pair which ensure fair operation, thereby avoiding starvation of components.
Lastly, \textit{ZeroMQ} guarantees atomic delivery of messages, which means that we can assume all parts of a message or none at all. This is the most desirable scenario for us since lost messages should be relatively insignificant, while malformed messages usually means strenuous error and edge case handling.

\subsubsection{Proxies}
In the interest of rapid iteration and ease of configuration, we deploy proxies at key points in the pipeline.
This allows us to add and remove components easily from the network, since components then do not need to know of each other --- they only need to know of the proxies.
An alternative would be to use some form of service discovery to allow components to discover and connect to each other directly.
We insert two proxies in the network: one between input collectors and the input processor and one between computation and publishers, as depicted in \vref{fig:impl_figure}.
Another benefit of using proxies is the ability to have a many-to-many connection, since the proxy binds both its frontend and backend socket, which components then connect to.

These proxies are written in C for performance and since they utilize a \textit{ZeroMQ} primitive for the actual forwarding, their operation is quite stable.
It should be noted that these proxies inherently introduce single points of failure in our beacon, which potentially could be detrimental for continuous and stable operations. In then next section (\vref{sec:proxy-avail}), we describe a way to mitigate this.

While the two proxies serve the same purpose, i.e.\ pass along traffic between components, their workloads are vastly different.
The proxy between computation and publishers is a \textit{forward} proxy, which transparently facilitates publish/subscribe pattern.
In our beacon, this proxy will never see a high frequency of messages, since the number of outputs, commits, and proofs, are limited by the beacon output frequency.
The proxy will, however, be subjected to significantly larger messages, due to especially commits which must contain every input used in the output computation. Even with the large message, the infrequency of them is far from saturating the capacity in this proxy.

Between the input collectors and input processor, we have a \textit{stream} proxy, which ensures that the fan-in and fan-out patterns are executed fairly, i.e.\ no connected components are starved.
This is facilitated through a round-robin message distribution.
In contrast to the previous, this proxy is required to be able to handle a substantial amount of messages, since every input submitted to the beacon will pass through it.
While the messages are remarkably smaller (64 bytes of application data), the amount of messages can become a problem, when the fair distribution and no starvation policy must be enforced.

In the later \vref{sub:bottlenecks}, we evaluate the performance of our proxies.

\subsubsection{High-Availability}\label{sec:proxy-avail}
One way to mitigate single points of failures could be to implement a pattern called \enquote{binary star} by \textit{ZeroMQ}.
Here a component is configured with two instances, a primary and a backup.
The backup can then take over and signal for a new backup to be started if the primary disappears from the network.
This pattern can potentially be applied to all components exposed as a single point of failure and would be sufficient in most cases of crashes.

Another measure to provide high-availability in the beacon is the delegation of user interaction to input collectors and publishers.
In the system, user interaction is the most demanding task regarding availability, and the statelessness of the components means that adding new instances is as easy as executing a shell command.
