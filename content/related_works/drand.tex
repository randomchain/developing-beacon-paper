\subsection{Drand - A Distributed Randomness Beacon Daemon}
\label{sub:drand_a_distributed_randomness_beacon_daemon}
The \gls{dedis} at EPFL in Switzerland, has developed an open source distributed randomness beacon called \textit{Drand}\footnote{\url{https://github.com/dedis/drand}}.
The beacon is related to the \textit{Specialized MPC} archetype, and deploys a set of limitations and assumptions, which makes it well-suited for a private setting.
Drand links nodes together to periodically and collectively produce what they claim is \enquote{publicly verifiable, unbiasable, unpredictable} random values.

At it's core, Drand consists of two phases, a setup phase, which requires knowledge of all participating nodes; and the randomness generation phase, which must be initiated by a single leader.
The setup phase and requirement for a leader to initiate the randomness generation, makes the operation of Drand static, i.e.\ new nodes cannot join an already running protocol.
However, due to the mechanisms underlying the randomness generation, faulty or unavailable nodes may not hurt the availability of the beacon, provided a defined threshold of running nodes is achieved.

\subsubsection{Setup Phase}
During the setup phase each node to be xincluded in the beacon, generates a public/private key-pair, to be used long-term.
A file called the \textit{group file} is then created, consisting of all participants' public keys, and configuration metadata regarding the beacon operation.

The \textit{group file} is then distributed amongst the nodes, whom then participate in a \gls{dkg} protocol.
This protocol creates a collective public key, and a sharded private key, with each node in possession of a unique shard --- which in turn is used for the internal cryptographic operations of Drand.

\subsubsection{Randomness Generation Phase}
Any node may function as a leader and initiate the randomness generation phase by broadcasting a message consisting of a time stamp to all nodes.
This time stamp message is then signed by all participating nodes with a threshold version of the \gls{bls} signature scheme.
The threshold version allows any node to construct the full signature, which is the random output, given that enough nodes has provided their shard signature.

Output values can be verified by using the public key generated in the distributed key generation.
