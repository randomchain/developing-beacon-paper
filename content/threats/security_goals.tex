\subsection{Security Goals - Probably needs a new name?}

Based on our security goals, we choose to focus on designing a beacon of the \emph{Transparent Authority} type, using \emph{User Input} to produce randomness. This matches our security goals the best, and allows us to be more precise in the assessment of threats to the beacon. In addition to the overarching security goal, we also have a set of security of properties we want our beacon to include:

\begin{description}
	\item [Trust] Users should only need to trust themselves. The output of the beacon must be impossible to bias against a user that has successfully contributed to it. \mtjnote{Change this to extremely hard --- not impossible}
    \item [Byzantine Behavior Detection] If the beacon deviates from protocol, it should be easily detectable.
    \item [Fault Isolation] If a component acts byzantine, the behavior should not propagate to the rest of the system.
    \item [Availability] The beacon should have a variety of input and output options for both usability and redundancy. \mtjnote{Is this really what we should call \enquote{Availability}?}
\end{description}
