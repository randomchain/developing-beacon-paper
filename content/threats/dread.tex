\subsection{Threat Analysis}
We consider a wide variety of threats, and analyze their severity according the DREAD framework with some modifications. DREAD consists of five metrics~\cite{dread}, that are scored on a scale of 1 to 3 (low, medium, high):
\begin{description}
    \item [Damage] How bad would such an attack be?
    \item [Reproducibility] How easy is it to reproduce such an attack?
    \item [Exploitability] How much work is required to launch the attack?
    \item [Affected Users] How many users will be impacted?
    \item [Discoverability] How easy is it to discover the threat?
\end{description}

\subimport{}{threat_table.tex}

However, the \emph{discoverability} can be considered to reward security through obscurity. As such we do not consider it, as an adversary with unlimited resources would naturally also know of all exploits.
%Were gonna need a pretty solution to DREAD structure
We consider threats to two sides of the beacon, the \emph{Availability} and \emph{Integrity}.

\subsubsection{Availability}

\newcommand{\parathreat}[1]{\paragraph{#1}\hspace{-1ex}}

\parathreat{Input Flooding}\drea{2,3,2,3,10} Adversaries can overwhelm the beacon with inputs to prevent other users from contributing their own input, thus denying service. Another approach could be perform a standard \gls{dos} attack on the central server of the beacon to prevent operation. This is quite a big threat, as users can temporarily be denied service, and the attack is quite easy to execute --- any determined adversary could rent a botnet to flood input collectors.    %D 2 - the beacon will be back, R 3 - anyone can borrow a botnet for dos attacks, and operator can pull the plug, E 2 - they need to use a dos tool,  A 3 - noone can use the beacon, but eclipse would only hit 1 user.
\parathreat{Shutdown}\drea{3,3,2,3,11} A malicious beacon operator can shut the beacon down, completely denying availability. This is a threat that is practically impossible to prevent for a beacon run by a single operator, as the operator can always shut any part of the beacon down. It is also trivial to execute for the operator, but requires that power.
%D 3 - complete shutdown, R 3 - easy to pull the plug, E 2, A 3 - affects all users. We cant prevent this attack,
\parathreat{Withholding Input}\drea{3,3,2,3,11} The operator can withhold outputs that are not favorable to his interests. This threat is also quite severe as it not only denies an output, but also ruins the beacons reputation. Even more gravely, it can even be blamed on technical mishaps such as crashes to conceal malicious behavior, while remaining easy for the operator to execute.
%D 3 - ruins output, R 3 - easy, E 2, A 3 - affects all users.
\parathreat{Eclipsing the Beacon}\drea{3,1,1,3,8} An adversary can deny all users from providing input or receiving the output. This is a quite small threat, as it is extremely hard to completely prevent a determined party from accessing the beacon, and such an eclipse would still only affect that party.
%D 3 - no use for all users, R 1 - extremely difficult to accomplish, E 1 - likewise, A 3 - entire system affected.
\parathreat{Eclipsing (Select) Users}\drea{2,1,1,1,5} An adversary can deny select users from accessing the beacon to provide input or receive input. This is a quite small threat, as it is extremely hard to completely prevent a determined party from accessing the beacon, and such an eclipse would still only affect that party.
%D 2 - no use for a single user, R 1 - extremely difficult to accomplish, E 1 - likewise, A 1 - only one user affected.


\subsubsection{Integrity}

\parathreat{Input Biasing}\drea{3,3,2,3,11} Adversaries can provide input that biases the output to their benefit --- typically in the form of Last Draw Attacks, where they provide the last input to the computation. This is a severe threat to the beacon, as it biases the output to the benefit of the adversary, and violates the unpredictability property of the beacon. The attack can be executed by anyone with access to the input collectors given that they have the ability to pre-compute outputs.
%D3 - Ruins the output for everyone, and breaks unbiased / unpredictable, R 3 - everyone can do it with a web browser, E 2 - requires the ability to precompute results, A 3 - all users affected
\parathreat{Output Degradation}\drea{3,3,3,3,12} As a variation of the previous attack, adversaries can supply bad input to reduce the quality of the output. This is also a large threat as it will affect the quality of randomness provided to all users, which may not even be useful. In addition, it is very easy to do given access to the input collectors, and could even happen by accident.
%D 3 - bad randomness, R 3 - easy to give bad input, E 3 - this could happen by accident, A 3
\parathreat{Input Manipulation}\drea{3,3,2,3,11} The operator can manipulate the input to bias the output of the beacon. He can also selectively exclude inputs from certain sources to deny them availability. This threat is quite severe, as the operator has direct access to manipulate the inputs, and may even be able to do so in a way that can not be detected. It is also quite easy for any operator capable of computing pre-images, and affects the randomness given to all users.
%D 3 - ruins output and beacon reputation, R 3 - easy life, E 2 - must be operator, A 3 - everyone is affected
\parathreat{Man in the Middle}\drea{3,1,1,3,8} Adversaries can intercept and change data sent between user and beacon. This threat is potentially very damaging, but also extremely hard to execute for adversaries. Due to the nature of beacons, we recommend using them when you need to agree on some random number --- thus, to intercept and manipulate inputs and outputs, the adversary would have to distribute the manipulated to all users, as they would otherwise disagree on the numbers, leading to the manipulation being discovered.
%D 3 - ruins beacons output, R 1 - needs to send to all users , E 1 - must be done to all users, A 3 - affects all users.
\parathreat{Emitting false Output}\drea{3,2,2,3,10} A malicious operator can output false results of the computation that benefit him. While this is technically a threat to the beacons integrity, the effects should be similar to those of a withholding attack. This is due to the fact that simply publishing false output would rapidly be discovered in a transparent authority beacon, making the input unusable.
%Equivalent to withholding attack, as users can verify that the output is false.
\parathreat{Leaking output}\drea{3,3,2,3,11} The operator can give access to the output earlier to some parties than others --- potentially selling early access. This threat can also be quite large, as we know nothing of how early access can be granted compared to when the randomness is used. It also violates the unpredictability property of the beacon, and is easily executable for any malicious operator of the beacon. Finally, in the worst cases, it would also affect all users.
%D 3 - beacon can precompute simple outputs, R 3 - easy to repeat, E 2, A 3 - worst case everyone is affected
\parathreat{Cryptography exploit}\drea{3,3,1,3,10} Weaknesses in the cryptography that allow adversaries to easily manipulate the beacon could exist or be found. This threat could fundamentally compromise the beacon and completely overturn any attempts at achieving security through cryptographic techniques. While we estimate it will be hard to find such exploits, they would likely be quite easy to apply once found, and would naturally affect all users. In this case one might also consider the effect quantum computers would have on the use of cryptography, which could also threaten the beacon.
%D 3 - adversaries gain extreme power to manipulate beacon, R 3 - easy to reproduce once found, E 1 - very hard to find, A 3 - beacon is compromised. Also consider quantum computers. They are the root of all evil.
