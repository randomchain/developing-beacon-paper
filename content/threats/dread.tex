\subsection{Threat Analysis}
We consider a wide variety of threats, and analyze their severity according the DREAD framework with some modifications. DREAD consists of five metrics~\cite{dread}, that are scored on a scale of 1 to 10:
\begin{description}
    \item [Damage] How bad would such an attack be?
    \item [Reproducibility] How easy is it to reproduce such an attack?
    \item [Exploitability] How much work is required to launch the attack?
    \item [Affected Users] How many users will be impacted?
    \item [Discoverability] How easy is it to discover the threat? 
\end{description}

However, the \emph{discoverability} can be considered to reward security through obscurity. As such we do not consider it, as an adversary with unlimited resources would naturally also know of all exploits. 
%Were gonna need a pretty solution to DREAD structure
\begin{description}
    \item [Input Biasing Attack] Adversaries can provide input that biases the output to their benefit. 
    \item [Denial of Service] Adversaries can overwhelm the beacon with inputs to prevent other users from contributing their own input, thus denying service. Another approach could be perform a standard \gls{dos} attack on the central server of the beacon to prevent operation, or performing an eclipse attack on certain users to deny them access. 
    \item [Hijacking] Attackers can attempt to seize control of the beacon, either by directly gaining access to the beacon and all accompanying operator privileges, or indirectly by corrupting the operator. 
    \item [Shutdown] A malicious beacon operator can shut the beacon down, completely denying availability.
    \item [Withholding Output] The operator can withhold outputs that are not favorable to his interests.
    \item [Selling Early Access] The operator can give access to the output earlier to some parties than others.
    \item [Input Manipulation] The operator can manipulate the input to bias the output of the beacon. He can also selectively exclude inputs from certain sources to deny them availability.

\end{description}
