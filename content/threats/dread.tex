\subsection{Threat Analysis}
We consider a wide variety of threats, and analyze their severity according the DREAD framework with some modifications. DREAD consists of five metrics~\cite{dread}, that are scored on a scale of 1 to 3 (low, medium, high):
\begin{description}
    \item [Damage] How bad would such an attack be?
    \item [Reproducibility] How easy is it to reproduce such an attack?
    \item [Exploitability] How much work is required to launch the attack?
    \item [Affected Users] How many users will be impacted?
    \item [Discoverability] How easy is it to discover the threat?
\end{description}

However, the \emph{discoverability} can be considered to reward security through obscurity. As such we do not consider it, as an adversary with unlimited resources would naturally also know of all exploits.
%Were gonna need a pretty solution to DREAD structure
\begin{description}
    \item [Input Biasing Attack] Adversaries can provide input that biases the output to their benefit --- typically in the form of Last Draw Attacks, where they provide the last input to the computation.
    %D3 - Ruins the output for everyone, and breaks unbiased / unpredictable - R 3, everyone can do it with a web browser - E 2, requires the ability to precompute results, A 3, all users affected. D - 3
    \item [Denial of Service] Adversaries can overwhelm the beacon with inputs to prevent other users from contributing their own input, thus denying service. Another approach could be perform a standard \gls{dos} attack on the central server of the beacon to prevent operation, or performing an eclipse attack on certain users to deny them access.
    %D 2, the beacon will be back - R 3, anyone can borrow a botnet for dos attacks, and operator can pull the plug - E 2, they need to use a dos tool. - A 3, noone can use the beacon, but eclipse would only hit 1 user.
    \item [Man in the Middle Attack] Adversaries can intercept and change data sent between user and beacon.
    %D 3 - ruins beacons output, R 1 needs to send to all users , E 1 must be done to all users, A 3 affects all users.
    \item [Hijacking] Adversaries can attempt to seize control of the beacon, either by directly gaining access to the beacon and all accompanying operator privileges, or indirectly by corrupting the operator.
    %Not a thing
    \item [Shutdown] A malicious beacon operator can shut the beacon down, completely denying availability.
    \item [Withholding Attack] The operator can withhold outputs that are not favorable to his interests.
    \item [Selling Early Access] The operator can give access to the output earlier to some parties than others.
    \item [Input Manipulation] The operator can manipulate the input to bias the output of the beacon. He can also selectively exclude inputs from certain sources to deny them availability.
    \item [False Output] A malicious operator can output false results of the computation that benefit him.
    \item [Exploitable Crypto] Weaknesses in the cryptography that allow adversaries to easily manipulate the beacon could exist.
    \item [Output Quality Attack] Adversaries can supply bad input and deny good to reduce the quality of the output.
    \item [Eclipse Attack] TODO

\end{description}

\mtjnote{Some text that ties this together with the next section.}
