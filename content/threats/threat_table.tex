\aboverulesep=0ex
\belowrulesep=0ex
\begin{figure}[htb]
\centering
\small
\newcommand{\tabletitle}{\sffamily\color{descriptionlabeling}\bfseries}
\begin{tabular}{@{}ll!{\color{clframe}\vrule}l@{}}
    \arrayrulecolor{clframe}

    & \multicolumn{1}{c}{\tabletitle Insider} & \multicolumn{1}{c}{\tabletitle Outsider} \\

    \settowidth\rotheadsize{\parbox{2.1cm}{\centering\tabletitle Attacks on availability}}
    \rothead{\parbox{2cm}{\centering\tabletitle Attacks on availability}} &
    \makecell[l]{Shutdown \\ Withholding output} &
    \makecell[l]{Input flooding \\ Eclipse beacon \\ Eclipse (select) users} \\

    \cmidrule(l){2-3}

    \settowidth\rotheadsize{\parbox{2.1cm}{\centering\tabletitle Attacks on integrity}}
    \rothead{\parbox{2cm}{\centering\tabletitle Attacks on integrity}} &
    \makecell[l]{Input manipulation \\ Leak output \\ Emit false output} &
    \makecell[l]{Input biasing \\ Output Degradation \\ Man in the middle \\ Cryptography exploit} \\


\end{tabular}
\caption{A categorization of identified attacks. Attacks target either availability or integrity of the beacon, and can be performed by either insiders (the beacon operator, or outsiders hijacking the beacon) or outsiders (users who are not inside the system). The insider may also execute outsider attacks.}\label{tab:attacks-categorization}
\end{figure}