\section{Beacon Definition}

\mtjnote{THIS IS DIRECTLY FROM PREVIOUS SEMESTER}

%A randomness beacon is a service that publishes random data at a known interval.
\enquote{Formally, let $B: f(I_t) \rightarrow O_t$, where $B$ is a beacon, $I_t$ is the input at time $t$, $O_t$ is the output at time $t$, and $f$ is a suitable function for transforming the input to the output.
For $B$ to be a beacon, it is run at a known, regular interval, $\delta$, such that $t+n\delta$ for any $n \in \mathbb{N}$ are valid output intervals for the beacon.
In \Vref{fig:abstract_beacon} an example of an abstract beacon can be seen.
The green area is what can be defined as the beacon, or \emph{beacon protocol} --- i.e.\ how input is collected, transformed to output and then published.}

\subimport{}{simple_beacon_fig.tex}

%A sufficiently general randomness beacon with enough entropy can be utilized for multiple use cases, and as such is of higher usefulness.
%Given the many use cases, we will therefore prioritize a use case agnostic beacon.

\subsection{Trust Assumptions}

We take the view of \enquote{Everybody, including the beacon operator, is secretly colluding against you and willing to put an unlimited amount of money towards manipulating or biasing the process for their own benefit}.

The only one you can trust is yourself, and this extends to the beacon. If you have not influenced the beacon output, it should be considered biased against you by default. In addition, even if you have influenced the output, given enough time resources to work around your influence, the output could still be biased against you. 
