\section{Requirements}%
\label{sec:beacon_requirements}
This section lists the requirements for a randomness beacon suitable for our security goals and the threats that exist towards beacons.
We decided on using the \emph{transparent authority} type of beacon in \cref{sub:delimitations}, which requires a high level of transparency, and as such we build requirements on top of that.

The requirements presented here will serve as a foundation of our design.

\subsection{Transparent Operation}
Users should be able to oversee that the beacon operates according to protocol and thus catch any deviations from it.
This in turn requires all aspects of the protocol to somehow publicly announce or display their work for users to verify.
Fundamentally, users should be able to see which inputs have been used to produce randomness and verify that they do produce the published output.
This benefits the integrity of the beacon as some previously mentioned attacks would be detected in this setting.
It is also a necessity for our chosen type of input, \emph{user input}, which requires users' ability to verify that their own input is used to produce outputs.

The first part of this is the input --- users should be able to see which inputs are used to produce an output of randomness.
Being able to verify whether their own input has been used allows users to determine whether they should trust the output.
If their input has not been used, they should not trust it.
Secondly, they should be able to repeat the process on their own computers as a means of verification.
This also requires the process to be deterministic which in turn removes the option for true randomness.
However, the output should still be unpredictable, even to the beacon operator.

\mtjnote{New requirement idea: \enquote{Secure}: Even if only one single participant is honest (and that can be you, thanks to Open Protocol.)}

\subsection{Open and Secure Protocol}
Anyone should be able to easily contribute to the beacon protocol to influence the random generation. There should be no requirements imposed on users to limit their contribution rate besides \gls{dos} protection. The protocol should be secure meaning that even if only a single participant is honest, the output is still unpredictable. This single participant will in the worst case be you.

\subsection{Timely Publishing}
The input, output, and any data needed for verification of an output should be enforced to be published as soon as possible to make the beacon more transparent.
By having a requirement of timeliness at the protocol level, we restrict the time a malicious operator has available to diverge from protocol before users will suspect them.

Giving users all the tools to replicate and oversee the process makes it difficult for adversaries to covertly manipulate the beacon to their benefit, and allows users to complete output computation themselves if the beacon stalls.
This in turn mitigates one of the greatest threats from the operator, input manipulation.
A beacon that does not reveal which inputs were used before publishing the output will essentially be admitting that they picked the inputs to bias the output.

\subsection{Practical}
A part of the goal is to create a beacon that is practical and implementable in the real world.
As such, we value requirements that other purely theoretical approaches may not consider.
Scalability of all components is important as we envision a general beacon suitable for many use cases.
Therefore, it should scale to at least several thousand users contributing with user input in every output.
Here, other approaches usually only focuses on scalability of the core theory, but we will consider all parts.
Furthermore, we value easy deployment and installation of the beacon, in order for it to not be a hindrance for deciding to run a beacon.

Likewise, it would be beneficial to allow different channels for input and output, both to make the beacon easier to access for users, but also to make it resilient to having any single channel attacked.
Should a single channel be attacked, input could still be submitted to another.
This requirement further increases usability since different users may prefer different input and output channels.
We also consider fault tolerance a valuable property to have, and having multiple channels still allows users to input if one fails.

% To summarize, we have the following requirements for our beacon:
% \paragraph{Transparent Operation} Users are able to oversee the protocol and see any deviation. This means that users are able to verify that the input is valid, i.e.\ their own input is included, and able to verify the output is valid, i.e.\ that the output is computed correctly from the input. Furthermore, the output is deterministic and unpredictable, which means that randomness is introduced because users are not able to reproduce or verify the randomness, but still, the output is unpredictable.
% \paragraph{Modular Input and Output Methods} Ability to include multiple input channels and multiple output channels provides redundancy in these areas.
% \paragraph{Timely Publication} The beacon is not able to significantly keep or delay output without users noticing.
% \paragraph{Fault Tolerance} The beacon is designed and implemented in such a way that a high fault tolerance is present.
