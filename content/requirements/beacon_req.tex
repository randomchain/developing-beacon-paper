\section{Randomness Beacon Requirements}
\label{sec:beacon_requirements}

This section will list the requirements for a randomness beacon, suitable for our security goals.
These requirements are separated into two groups: functional and non-functional requirements, as suggested by \citet[sec.\ 1.4.1]{swebok}.
In subsequent sections, we will perform a preliminary analysis of these requirements to reveal possible security threats, and design a beacon that fits these requirements and circumvents as many as possible of the found threats.

\subsection{Functional Requirements}
\label{sub:functional_requirements}
\begin{description}
    \item [Transparency]
Users should be able to observe which inputs have been used to produce a random output. This is important because it allows them to find their own input, and be able to trust an output, and contributes to both the \emph{Trust} and \emph{Byzantine Behavior Detection} goal.

    \item[Verifiable]
The output should be verifiable for any user, i.e.\ anyone must be able to verify that the beacon protocol produced the published output from the published input. This contributes to the \emph{Byzantine Behavior Detection} and \emph{Trust} goals.

    \item[Deterministic Output]
As a function of the previous requirement, the beacon should deterministically compute an output from a set of inputs.
As previously argued, this is because using true randomness will not allow users to reproduce the output.

    \item[Unpredictable Output]
The output should be unpredictable before the computation is done, even for the beacon operator.

    \item[Timely Output]
The beacon should publish the random output immediately after the computation is done, and any data necessary for the verification should be available to the user, contributing to the \emph{Byzantine Behavior Detection} goal.

    \item[Modular Input and Output]
The input and output channels of the randomness beacon should be extensible or configurable by beacon operators, to satisfy the \emph{Availability} goal.
This will allow a beacon operator to use multiple input sources and output channels.
For example, a beacon operator may allow users to provide inputs through APIs, email, and even tweets.
Similarly, the output may be tweeted, published on a website, and will subsequently be harder to prevent from being published.


%Strictly, it is not possible to be truly random, however for the vast majority of use cases this pseudo-randomness is good enough.
% Far more important is the following requirement.

    %\item[Commitment to Input] The beacon should produce a commitment to a set of inputs and publish the commitment as soon as possible after the input collection deadline, but before initiating output computation.
% This ensures that the beacon operator cannot add, remove, or order the inputs to his own benefit. \mtjnote{Maybe remove this from here --- this is a solution, not requirement}


\item [Internal Sanity \& Reasonability Checks]
The beacon should have both sanity and reasonability checks to detect as much byzantine behavior as possible, as stated in the \emph{Byzantine Behavior Detection} goal, and prevent it from spreading to other components as per the \emph{Fault Isolation} goal.

\end{description}

\subsection{Non-Functional Requirements}
\label{sub:non_functional_requirements}

\begin{description}
    \item[Fault Tolerance]
The beacon architecture should be designed so as not to have a single point of failure, i.e.\ it should be configurable to be fault-tolerant by utilizing redundancy, as a way to satisfy the \emph{Fault Isolation} and \emph{Availability} goals.

    % REDACTED \item[Something something] Operating the beacon should be incentivized and distributable among multiple participants. \msmnote{e.g.\ input sourcing by some, computation by another, etc.} \mtjnote{Why are these two necessary?} \mtjnote{This is two requirements, no?}

    \item[Usability]
It should be easy for users to contribute with input, and to fetch outputs, which in turn contributes to the \emph{Availability} goal.
In practice this should not be a major problem, since input can be contributed over an insecure channel (they are not secret), and likewise the output can be retrieved over an insecure channel because the correctness of computation can be verified.
\end{description}

\subsection{Non-requirements}
\mtjnote{We probably do not need this section.
Take inspiration from this text and incorporate the pieces somewhere else}
This section lists some requirements which may at first seem beneficial, but actually do not matter.

\paragraph{Signing output with signature:}
\mtjnote{Do we want to sign the output with our own signature? We probably do not need it for tamper-detection, as clients using the random number should verify the computation and the presence of their own input.}

\paragraph{Secure channels for input and output:}
\mtjnote{Inputs and outputs are not secret.
What if an adversary has complete knowledge of all inputs? Then he can only do it as fast as the original operator.
But he cannot change anything besides the order of inputs --- and he needs to commit to the order before computing.
What about fiddling with the output? Users can verify the computation AND should not use an output for which they have not seen the corresponding commitment to AND they should verify that their own number has been part of the computation.
So everything is good.}

\mtjnote{The above note mentions \enquote{commitments}, but we have not introduced these yet}
