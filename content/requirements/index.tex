\section{Beacon Requirements}
\label{sec:beacon_requirements}

Given the scenarios where simple but insecure protocols can be made secure by utilizing a randomness beacon, we present the following requirements for said beacon.
Due to the universal nature of these insecure protocols, we prioritize a use case agnostic beacon, which will fit multiple use cases.
This means that some requirements may be more relaxed in our case, compared to a specialized beacon, suited for one use case only.

The requirements of our randomness beacon are separated into two groups: functional and non-functional requirements, as defined by Ian Sommerville in \textit{Software Engineering}\msmnote{cite here}.

\subsection{Functional Requirements}
\label{sub:functional_requirements}
\begin{enumberate}
\item The beacon should accept input from and output to multiple sources, in an extendable and configurable manner.
\item The beacon should be able to deterministically compute a random output from a given set of inputs.
\item The beacon should commit to a set of inputs before initiating output computation, and the commitment should be published as proof.
\item The output should be unpredictable before the computation is done, even for the beacon operator.
\item The beacon should publish the output immediately after the computation is done, alongside any important meta data about the computation, e.g.\ parameters and witnesses to be used in verification.
\item The output should be verifiable for any user, i.e.\ anyone should be able to verify the authenticity of both the computation and origin of the output.
\end{enumberate}


\subsection{Non-Functional Requirements}
\label{sub:non_functional_requirements}

\begin{enumberate}
\item The randomness beacon should be extendable or configurable by beacon operators, i.e.\ it should be highly modular in nature.
      This will allow for customization of the beacon to embrace different use cases.
\item The beacon should not contain a single point of failure, i.e.\ it should be configurable to be fault-tolerant by utilizing redundancy.
\item Operating the beacon should be incentivized and distributable among multiple participants. \msmnote{e.g.\ input sourcing by some, computation by another, etc.}
\item It should be easy for users to contribute with input, and to fetch outputs.
\end{enumberate}



