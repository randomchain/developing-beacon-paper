\subsubsection{Sloth Parameters}%
\label{ssub:sloth_parameters}

The computation and verification time of the delay functions we use, Sloth, can be configured by adjusting two parameters.
These are the number of bits in the prime used and number of square root permutations, i.e.\ iterations.

Something something dark side.

\begin{figure*}
    \centering
        \footnotesize
    \begin{subfigure}[t]{0.4\textwidth}
        \centering
        \begin{tikzpicture}
        \begin{axis}[%
        width=1.075\textwidth,
        xlabel=bits,
        xtick={1024,2048,3072,4096},
        ymode=log,
        log basis y={10},
        yticklabel pos=right,
        legend pos=north west,
        grid=major]
        \addplot[scatter, shader=interp] table[y={COMPUTATION}, x={BITS}, col sep=comma, restrict expr to domain={\thisrow{ITERATIONS}}{9500:10500}] {sloth_data.csv};
        \addplot[scatter, mark=square, shader=interp] table[y={VERIFICATION}, x={BITS}, col sep=comma, restrict expr to domain={\thisrow{ITERATIONS}}{9500:10500}] {sloth_data.csv};
        \legend{Computation,Verification};
    \end{axis}
\end{tikzpicture}
\caption{Computation and verification time at 75000 iterations.}
    \end{subfigure}%
    \hfill
    \begin{subfigure}[t]{0.55\textwidth}
        \centering
        \begin{tikzpicture}
        \begin{axis}[%
            width=0.9\textwidth,
            xlabel=iterations,
            ylabel=bits,
            ytick={1024,2048,3072,4096},
            zlabel=time (s),
            grid=major,
            x dir=reverse]
        \addplot3 [only marks, scatter, shader=interp, mark=cube, mark size=3] table[y={BITS}, x={ITERATIONS}, z={VERIFICATION}, col sep=comma] {sloth_data.csv};
        \addplot3+[only marks, scatter, shader=interp, mark=*] table[y={BITS}, x={ITERATIONS}, z={COMPUTATION}, col sep=comma] {sloth_data.csv};
    \legend{Verification,Computation};
    \end{axis}
\end{tikzpicture}
\caption{Correlation between bits and iterations in relation to time of both computation and verification.}
    \end{subfigure}%
\end{figure*}
