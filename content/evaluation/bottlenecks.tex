\subsection{Bottlenecks}%
\label{sub:bottlenecks}

We examine two potential bottlenecks in our beacon.
These are components which require the most effort to scale horizontally, and as such for simplicity we want to discover the current limits before scaling.
Furthermore, it is important to determine if these are actual bottlenecks before attempting to scale, as we want to avoid premature optimizations.

\subsubsection{Proxies}%
\label{ssub:proxies}
Proxy performance test here.
Probably reasonable results.


\subsubsection{Input Processor --- Building Merkle Trees}%
\label{ssub:input_processor_building_merkle_trees}
The most expensive task performed in a bottleneck, is building the Merkle tree in our input processor.
This task is done periodically when it is time to compute a new random output.
It should not take a significant amount of time, since this would extend the time between the last seen input and publishing the commit.
As such, we examine how the number of leaves, i.e.\ inputs, relates to the time it takes to build the Merkle tree.

In \cref{fig:merkle_build}, a linear growth in build time is seen as a factor of the number of leaves.
The growth is slow and is negligible in our beacon. There needs to be well over 2 million leaves to result in a build time over three seconds.
We can describe the relationship as follows, where $N$ is the number of leaves:
$$
1.41\,\mu\text{s} \cdot N + 1.41\,\text{ms}
$$

Admittedly, the build time could be a problem if significantly many inputs are used.
However, in this case one might reimplement the input processor in a more performant language than Python, e.g.\ C.
In addition, the construction of Merkle trees is trivially parallelized.
Our implementation of Merkle trees does not take advantage of this fact, and so building subtrees in multiple processes and merging them to form the final tree will likely provide a significant speed-up with a factor close to the number of available CPU cores.

\begin{figure}
    \centering\footnotesize
    \begin{tikzpicture}
        \begin{axis} [
            width=\axisdefaultwidth,
            height=5cm,
            xlabel=number of leaves,
            ylabel=time (s),
            legend pos=north west,
            ymajorgrids=true,
            xmajorgrids=true,
            major tick length=0pt
            ]
            \addplot [only marks, mark=*, mark size=1, color=GoogleBlue] table [col sep=comma, x index=0, y index=1] {plotdata/merkledata2.csv};
            \addplot [no markers, color=GoogleBlue] table [col sep=comma, x index=0, y={create col/linear regression={y=BUILD}}] {plotdata/merkledata2.csv};
            \addlegendentry{%
        $\pgfmathprintnumber{\pgfplotstableregressiona} \cdot x
        \pgfmathprintnumber[print sign]{\pgfplotstableregressionb}$}
        \end{axis}
    \end{tikzpicture}
    \caption{Correlation between number of leaves and the time it takes to build a Merkle tree, with those leaves.}%
    \label{fig:merkle_build}
\end{figure}
