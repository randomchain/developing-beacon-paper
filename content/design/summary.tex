\subsection{Summary}
In the process of designing of designing our beacon, we have considered the threats to a beacon and designed measures to mitigate them. We present the threats we consider to be successfully mitigated with our design.

\parathreat{Output Degradation}\drea{2,3,3,2,10}
The quality of the output can not be lowered by the submission of bad input, as we use hashing functions. The quality of their output is upper bounded by the best input, but not lower bounded by the worst.

\parathreat{Input Manipulation and Biasing}\drea{3,3,2,3,11}
We have designed the beacon around using delay functions specifically to prevent these kinds of attacks. As previously mentioned, they require an adversary, insider or not, to spend a significant amount of time to compute a single pre-image, which is essential to this type of attack. In addition we have provided a comprehensive exploration of when to trust the beacon, which makes users even more secure against this type of attack.
This also makes any attempt at this attack from the operator equivalent to a withholding attack, as users should not use an output they did not see a timely commit for.

\parathreat{Leaking Output}\drea{3,3,2,3,11}
Our delay function based CCO workflow also prevents the operator from leaking outputs that give any significant advantage. An output will never be used unless a commit for it is seen, and the commit contains all of the data required to compute the output alongside the operator - thus the operator can only leak outputs that are already pre-determined, removing their 'market value'.

With this we have covered some of the largest threats to the beacons integrity, making the beacon significantly more secure.
