\subsection{Probabilistic Trust}%
\label{sub:probabilistic_trust}
In our approach to a randomness beacon we want to push beyond the need for honest operators and naïve users.
To achieve this we first need to quantify trusting the beacon and then determine thresholds for reasonable behavior.
This will effectively provide a measure of probabilistic trust, where users decide for themselves if the probability of honest operation is adequate.

We present a property which, if satisfied, means a user can trust that the beacon operator is not capable of fooling them.
This property is true if the user determines that nobody is able to compute the delay function in the time between the users input and the user receiving the beacon's commitment.
This can be condensed to:
\begin{equation*}
    t_\text{COMMITMENT} - t_\text{INPUT}\enspace <\enspace T_\text{DELAY FUNCTION}
\end{equation*}
given that $t_\text{INPUT}$ is the time when the user sent the input, $t_\text{COMMITMENT}$ is when the user received the commitment, and $T_\text{DELAY FUNCTION}$ is the fastest computation of the delay function.
This means that for users to be more likely to trust a beacon, the time between sending the input and receiving the commitment must be significantly smaller than the time between the commitment and the output.
In fact, it must be smaller than the shortest time the user thinks the operator could compute the delay function.

An example could be a that user that can compute the delay function in 10 minutes believes that the world's fastest computer can do so 5 times faster, i.e.\ in 2 minutes. In this case the user should trust the output if he sees a commit to a set of inputs containing his input within 2 minutes of his input.
This relation between the time taken to compute the delay function and the time before a commit is seen allows users to flexibly adjust, when they are willing to trust the outcome has not been biased against them.

Taking this into consideration we present a beacon operation protocol which can be adjusted to increase or decrease this limit.
The operation must be sequential which means that we must collect input before computing the delay function.
However, because we want to spend more time computing than we are collecting input, a strictly sequential beacon will contain dead spots where no user is submitting input.
This may be acceptable in some scenarios, but we want to design a beacon which always accepts inputs and will not be suspected of malicious operation.
To achieve this we parallelize the beacon protocol, meaning that several delay functions run in parallel but offset in time and on different input.

In~\cref{fig:beacon_parallel_timeline} this is illustrated, where these offset but parallel beacons are seen.
We observe that no input collection is run in parallel nor overlapping, which resembles a constant stream of input collection.
Moreover, the computation components can eventually be reused for future beacon computations, thereby eliminating the need for spinning up new computation services.
These observations are depicted in \cref{fig:beacon_parallel_timeline_real}, where the beacon would output at each circle shown in the diagram.

\subimport{}{timeline_diagrams.tex}

\subsection{Scalability}
Individual services in the \gls{soa} can be scaled as needed.
Parallelizing a pipeline to increase the output frequency is also possible.

In some scenarios a beacon may consist of multiple computation services as a mean of redundancy, as long as each computation is run on the same input.
The same can be said about input processors, where a beacon may need redundancy to likewise avoid a single point of failure.
This presents the issue of consensus about which input to use, but that is out of scope for this report.

\subsection{Threats}
\mtjnote{A section that evaluates which threats have been solved}