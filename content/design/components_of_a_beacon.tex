\subsection{Services}%
\label{sub:components_of_a_beacon}

An instance of a randomness beacon designed in the \gls{soa} pattern will consist of a number of services.
To fulfill our requirement of modular input and output methods, we will allow multiple different input collectors and output publishers to exist in the system. \citet{randomzoo} mentions having several input sources, but to our knowledge we are the first to generalize it in the design to allow virtually any source.

A number of \textsc{input collector} services collect input from a myriad of different sources. These sources could for example be email, irc, a bot for your favorite instant messaging service, tweets with a specific hashtag, raw TCP or HTTP connections, a pretty website, SMS, Morse telegraph, or carrier pigeon\footnote{For carrier pigeon we recommend the \emph{IP over Avian Carriers} proposal: \url{https://www.ietf.org/rfc/rfc1149.txt}}.
An \textsc{input processor} service acts as an aggregator of the input from all collectors and hands it over to the \textsc{computation} service, which commits to the aggregated input and runs the computation to generate a pseudorandom output.
Finally, various \textsc{publisher} services publish the commitment, output, and any relevant proofs to different outlets. These outlets could be the same media as input collectors use but can be different.

An simple randomness beacon illustrating these services and their relationships can be seen in \cref{fig:beacon_arch}.

\subimport{}{abstract_architecture_diagram.tex}

Being a loosely coupled system, the arrows in the figure are not a given. Services need a way to know about other services. As such, some kind of service discovery is needed in the implementation. Service discovery can be a single point of failure but is mitigated by using redundancy~\cite{soa_redundancy}.
