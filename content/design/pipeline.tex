\subsection{Pipeline}%
\label{sub:pipeline}
In our beacon, as illustrated in \cref{fig:beacon_arch}, data only flows one way through the system and each component performs an independent transformation.
This effectively means that a beacon is a \emph{pipeline} where data flows into the system as inputs, is processed, transformed to a random output, and lastly published.

The pipeline architecture can be seen as a specialization of the \gls{soa} pattern, since each step in the pipeline is a service as defined in a \gls{soa}.
When considering the system as a pipeline some restrictions are imposed compared to the more generic \gls{soa}, because data only flows in one direction.
This means that if data is lost due to failure, e.g.\ a component crashing, we cannot inform previous pipeline components to resend the data.
However, since our beacon design is meant to operate in a forward-only manner, this loss of data should be tolerated, and even expected in some cases.
This underlines the fact that users should always verify both inclusion of their input and correct computation of output --- users should not assume that a submitted input is always included in the next output.

