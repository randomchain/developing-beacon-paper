\section{Introduction}
\emph{Randomness beacons}, services emitting unpredictable random values at fixed intervals, have been around for a long time.
In 1983 Michael O.\ Rabin coined the term and used one to add probabilistic security in several protocols~\cite{rabin1983transaction}.
In this definition, a randomness beacon was to be seen as an impartial third-party trusted to be unbiased towards any outcome.
As such, Rabin's beacon is to be \emph{trusted}, i.e.\ you should trust the beacon operator (the entity running the beacon service) to not be biased because you cannot verify that they are unbiased.

For quite a while, randomness beacons did not receive much attention, probably because alternatives to Rabin's protocols not requiring a trusted beacon were used instead (such as~\cite{BGMR}).
Circa 2010 a renewed interest in beacons was seen as an increase in beacon-related literature, and the trend in this new literature was to remedy the need to trust the beacon operator.
We believe it may have been a reaction to revelations like the NSA whistle-blower leaks that diminished people's trust in authorities.
In other words, people had their eyes opened to the fact that \emph{trust} can be an issue in itself and that removing the need for trust in any one entity could, in some cases, be beneficial.
As an example, cryptocurrencies have flourished in recent years alongside a sharp rise in the popularity of blockchains --- two technologies seeking to facilitate cooperation of mutually distrustful users.

Conceptually, randomness beacons seem to fit this environment of minimizing the need to trust as a randomness beacon is acting as an impartial party.
However, the \enquote{old} way, requiring users to trust the operator, simply shifts the trust issue to the centralized entity, i.e.\ the beacon operator.
% If anyone uses a randomness beacon, it is specifically because they do not wish to trust anyone.
% Therefore, we believe trusting the randomness beacon is a moot point.
The merit of a randomness beacon lies in contexts where a set of users needs to agree on some random outcome, but do not trust each other or any single party to make the decision.
Therefore, a randomness beacon is \emph{not} required in the case a user needs randomness for just himself. In this case, standard ways of generating randomness on a computer are far easier, and probably provide \enquote{better} randomness.
Similarly, if users trust each other or any other party, the randomness generation is also trivial.
A randomness beacon is not necessarily generating better randomness --- it merely allows users to agree on the same randomness without trusting anyone.
%The body of this work will be to discuss, design, and implement a beacon following the concept of not requiring users to trust it.% Before doing so, let's take a small detour and talk about potential use cases and the motivation for using a beacon at all.

Even though the literature theoretically argues for a particular solution, we have not seen many implementations of them. There is a bridge from theoretical solution to an actual running beacon, and we believe this bridge is uncharted territory. The literature only lightly touches the subject. We want to touch this subject, and find out which implications appear underway.
\mtjnote{If we have not seen this bridge (but have seen theory that is good), why do we need to invent our own theory and then implement it? Why not just implement some of the prior literature?. We need to write that we think we are smarter than all the prior literature.}

\subsection{Terminology}
In this work, we use the terms \enquote{randomness beacon} and \enquote{beacon} interchangeably. %, and make no initial assumptions on the properties of a beacon.
As discussed, we see two distinct groups of randomness beacons: the ones requiring users to blindly trust the beacon operator (à la Rabin's original beacon) --- we refer to these as \emph{trust-requiring} beacons; and the \enquote{new generation} randomness beacons of recent literature, which convincingly prove to all users of the beacon that nothing dubious happened during the creation of the random output --- we refer to these as \emph{trust-minimizing} beacons.

%In this work, we will use the terms \enquote{randomness beacon} and \enquote{beacon} interchangeably, and make no initial assumptions on the properties of a beacon. We also use \enquote{trust-requiring} and \enquote{trust minimizing} to refer to the generations of beacons.

\subsection{Security Goals}\label{sec:security_goals}
A trust-requiring beacon simply shifts the issue of trust. Therefore, we strive to design and implement a trust-minimizing beacon that works on the most pessimistic assumption possible: Everybody, including the beacon operator, is secretly colluding against you and is willing to put an unlimited amount of money and resources towards manipulating or biasing the randomness. As such, you can only trust yourself.

We realize it may not be possible to provide a bulletproof solution under these assumptions. However, these assumptions describe the mindset we take on while designing and implementing the beacon. For the actual solution, we relax the trust to \emph{probabilistic trust} where each user individually decides how much they desire to trust.
In essence, a user can know with some very high probability that the randomness has not been manipulated.

\subsection{Beacon Context}
As stated, beacons are relevant in contexts where several users want to agree on some random outcome, but do not trust anyone to solely decide that outcome.
This pattern might be a niche, but many use cases bear resemblances to it.

Consider the generic use case of sampling. Essentially, sampling is about selecting representative data points, potentially with high stake consequences. It would not be far-fetched to imagine someone wanting to bias this sampling process in order to skew the results. A very obvious sampling process is lotteries, which need to randomly sample winners.

The field of cryptography also contains use cases.
Many cryptographic schemes require some constants in the design, and it has been shown that some schemes have been intentionally built with specific constants in order to facilitate a backdoor~\cite{nist2014backdoor}.
Selecting constants with a randomness beacon can prove to the users of such cryptographic schemes that the constants were not manipulated and thus are unlikely to contain backdoors~\cite{baigneres2015trap}.

Staying with the cryptography, \gls{zksnark} require a lengthy process of bootstrapping in the very beginning. Traditionally, this bootstrapping has been performed by a \gls{mpc}. However, \gls{mpc} scales poorly because of the amount of communication between parties. \citet{mpcsnarks} suggest avoiding the \gls{mpc} and instead having users directly contribute a random number. To avoid a last-draw attack from the last user contributing, they apply an output from a randomness beacon. Thus a lengthy \gls{mpc} is substituted with a quick round of user input and sealing the deal with a randomness beacon.

\begin{comment}
\mtjnote{Rewrite so that we describe a generic use case: Wanting randomness that several people can agree on, but they do not trust each other.}

It is useful to consider what type of users the beacon may be useful to. In general, the beacon can be by any type of multitude that desire to use transparent randomness \mtjnote{wat}.
The best fit for this is trustless applications where parties can be expected to cheat if it would benefit them.

A beacon is not needed for any single party, as they could simply generate their randomness themselves.
It is the need for multiple parties to agree on something random that creates the need for public randomness.

In addition, it is not needed for a setting with trust between the parties, as they could then simply trust each other to generate randomness.
In trust-less settings this is not an option, and so users need randomness they can trust not to be biased, or at least detect if it is.
This is where our beacon finds its niche of applications.
%In which environments are beacons suitable? Users of the beacon output may be considered private individuals, scientists, companies, and corporations.
\end{comment}
%As a side note, all examples of use cases mentioned in this section also work with a trust-requiring beacon, \emph{if} it is reasonable to assume users are willing to trust the beacon.

%While we like the concept a randomness beacon offers (a tool to remove trust), we struggle to follow many of the use cases proposed in the literature. Classic use cases include

%\mtjnote{Write about zk-snarks}

%This recent literature presents some promising ways of creating an unbiasable beacon. However, there still exists no \enquote{killer} \enquote{real-life} use cases of a randomness beacon.
%We see many potential use cases, and present a variety of them with roots in the literature, before constructing a beacon of our own and performing a security analysis on it.

\subsection{Beacon Concepts}

A randomness beacon emits an unpredictable random value at a regular interval, e.g.\ every five minutes. \cref{fig:abstract_beacon} shows the workings of a simple, generic beacon. The beacon performs \emph{some} computation on an input source. The result of the computation is the output, which is sent to users. This workflow is repeated indefinitely at the specified interval.
\subimport{}{simple_beacon_fig.tex}

Examining existing beacons, a few common ways of composing the input sourcing and computation are discernible.
These can be described as specific models.
Here, we distinguish between an input source model and operational model. The input source model describes the way the beacon sources its input, and the operational model describes the design of the protocol, i.e.\ how to perform the computation and publish the output. These models are based on our earlier work \fxnote{Insert reference to own work previous semester?}.

\subsubsection{Input Source Models}
\mtjnote{Cite outselves} describe three sources of input:

\paragraph{Private Input Sources} A beacon can use some private source of data to to produce randomness. This allows them to obtain produce randomness of high quality at a high rate, but requires users of the beacon to trust the beacon and its randomness.
As we argued, true randomness as input cannot be trusted in our setting, since it cannot reliably be distinguished from carefully crafted values that appear to be random.
An example of this is the \gls{nist}~\cite{nistbeacon} randomness beacon that observes quantum mechanical effects to produce high-quality randomness.
Ultimately it requires trust, since the observations cannot be repeated, and therefore users cannot make sure that the value is indeed from observing the quantum mechanical effect.
As such, the users need to blindly trust the beacon operator, which in the case of \gls{nist} can be hard given their history~\cite{nytimes-nsabackdoors, nytimes-nsaconstants, nist2014backdoor}.

%Fortunately, the vast majority of use cases do not actually require true randomness.
%Far more desirable is the side effect of producing randomness: \emph{unpredictability}.
%As long as the output is unpredictable for all parties including the beacon operator, we do not really need true randomness, and the beacon output can be produced by a deterministic algorithm.
%And if implemented correctly, deterministic pseudo-randomness is just as good for virtually all use cases and provides a good level of unpredictability.
%No true randomness is necessary but we care deeply about the unpredictability from deterministic pseudo-randomness.

\paragraph{Publicly Available Sources} Using a publicly available source that everyone can agree on the value of, such as bitcoin transaction hashes, stock market data or lottery winning numbers from several international lotteries.
The user must trust the source, and this is reasonable because these sources are governed by some guarantees. E.g.\ in case of bitcoin, the transaction hashes have a monetary value and they are hard to pre-image. The rate of the source also dictates that of the beacon which can be an issue for some use cases. Users also have to interact with the source to indirectly influence the beacon and prevent collusion. However, it may also be harder for colluding adversaries to bias the beacon through the source unless they are in complete control of it.

\paragraph{User Input}
A user can be allowed to directly provide input to the computation.
The idea is that a user provides a value that they firmly believe is so random that all other users cannot reasonably generate the same value.
The beacon then performs an operation on a set of user-supplied input, where each input is a value that a specific user believes is a sufficiently random. The output of the beacon is structured in a way that
\begin{eletterate*}
    \item allows all users to verify the inclusion of their value and
    \item allows all users to verify the validity of the computation.
\end{eletterate*}

If these are satisfied, the user knows that a value they trust to be random has been part of the random output generation. The computation performed by the beacon should ensure that users cannot knowingly bias the output to anyone's disadvantage. As such, the user knows that his input was not knowingly \enquote{counteracted} by another used.

\subsubsection{Operational Models}
\mtjnote{cite outselves} identify three ways in which a beacon is typically operated:

\paragraph{Autocratic Collector} A beacon is run by a party, which deems it irrelevant to prove honesty. Instead, they require users to trust them to do what they say. As such, the computation is a black box with no possibility for proof of honesty. This type always lies in the trust-requiring category of beacons, i.e.\ the old generation.

\paragraph{Specialized \acrshort{mpc}} Users utilize \acrfull{mpc} to collectively produce randomness, typically from their own inputs. Given an honest majority, this type of beacon produces randomness that is not biased against the participants, and although work has been done in the field, they are difficult to scale to large groups~\cite{cascudo2017scrape, syta2017scalable}.

\paragraph{Transparent Authority} A single entity collects input and publishes it with a focus on transparency. Users can by observing the beacon verify that the beacon behaves according to protocol. This does not directly prevent byzantine behavior, but rather makes it difficult or nearly impossible to hide such behavior. This type also support a wide variety of implementations, and is scalable to a public setting.

\subsection{Delimitations}%
\label{sub:delimitations}
We want to create a randomness beacon that is secure under the assumption that everyone may be colluding against a given user, as per our security goals in \cref{sec:security_goals}. As such, we can immediately see that the autocratic collector is not suitable for our security goal, because it requires users to trust its hopefully honest operation. The \gls{mpc} model does not scale well enough for general use. Further, this model assumes an honest majority, which is a weaker assumption than our security goals. Therefore, it is not be suitable for us. This leaves us with the transparent authority model, which we adopt.

Regarding input source models, we can immediately discard private input sources. They are tied to the autocratic collector model, and as such do not work for the transparent authority. The guarantees by publicly available sources are weak compared to user input. If sufficiently paranoid, the user will want to bias these publicly available sources to make sure all other users are not colluding. Therefore, user input is the simplest solution and provides the strongest guarantees for the user.

\subsection{Contributions}
We bridge the gap between the literature and the real world by designing and implementing a secure, trust-minimizing randomness beacon based on the transparent authority model with user input. To this end, we conduct a structured analysis of the threats to a randomness beacon, and design a beacon based on these threats. We design the beacon from the ground up based on tried and tested methods as well as novel ideas. Specifically, we differ from previous transparent authority approaches by describing a way of parallelizing the beacon pipeline, we provide a thorough analysis of trust assumptions \mtjnote{what do we call it?}, and further we minimize the consequences of an output withholding attack performed by the beacon operator. Combined, our novel ideas decreases the need for trust and limits the severity of attacks.
