\section{Introduction}

In our modern society identification and authorization, by means of digital solutions, has become a central part of everyday life.
From unlocking our front door to verifying our payments at the local grocery store, an significant number of people are using so called smart cards, to facilitate the underlying authorization.
The most prominent type of smart cards are Java Cards, which leverage a subset of Java, used to develop applets capable of executing secure and secret protocols on the smart card.

Authorization is a process with varying degrees of critical operation.
Some use cases require more timely processing, while other might sacrifice speed for an extra round of confirmation.
Common for all application of authorization is the requirement of no false positives, i.e.\ nobody should be able to gain access to something they are not authorized to.
This obviously becomes a problem, in scenarios where the smart card is the only factor used in the authorization process, since a stolen smart card has no way of knowing who is using it.
To mitigate this, smart card authorization is often supported by pin codes or passwords.

\fxnote{Chosen cyphertext attacks, might be able to extract private keys with certain protocols}

\fxnote{Not possible in computationally expensive algorithms}
\fxnote{But expensive algorithms equals expensive smart cards}


