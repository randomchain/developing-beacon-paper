\subsubsection{Proofs of Delay}%
\label{ssub:proofs_of_delay}

As an extension to the ideas presented by \citet{randomzoo}, \citet{bunz2017proofsof} use a delay function and the \gls{btc} blockchain as a public available source.
They apply the delay function to mitigate issues of biasing the blockchain in anyones favor, and to limit the benefits of a block-withholding attack.
These two attacks are both claimed to be prevalent in naïve blockchain based randomness beacons without usage of a delay function.

The operation of the beacon is based on operator election, with possibility for anyone to become the new operator.
Outputs can be publicly contested, hereby prompting the operator to verify correct execution.
They present an incentive structure for operating the beacon, and fulfilling verification, which essentially relies on the beacon being operated as a \enquote{greater good}.
The contesting and verification is implemented in an Ethereum smart contract, which attaches a cost to contesting correct operation.
Moreover, the usage of a smart contract limits the availability for users not invested in the Ethereum blockchain, and restrict the possible algorithm usage in the delay function to sequential hash chains.

