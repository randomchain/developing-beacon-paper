\subsection{Beacon Context}
As stated, beacons are relevant in contexts where several users want to agree on some random outcome, but do not trust anyone to solely decide that outcome --- a pattern which fits a number of use cases.

Consider the generic use case of sampling.
Essentially, sampling is about selecting representative data points, potentially with high stake consequences.
It would not be far-fetched to imagine someone wanting to bias this sampling process in order to skew the results.
One such sampling process is lotteries, which need to randomly sample a pool of participants to draw winners.

The field of cryptography also contains use cases.
Many cryptographic schemes require some constants in the design of algorithms, and it has been shown that some schemes have been intentionally built with specific constants in order to facilitate a backdoor~\cite{nist2014backdoor}.
Selecting constants with a randomness beacon can prove to the users of such cryptographic schemes that the constants were not manipulated and thus are unlikely to contain backdoors~\cite{baigneres2015trap}.
One could even in some use cases imagine a \enquote{refreshing} algorithm where constants are variables which change with new beacon outputs.

Staying in the field of cryptography, \glspl{zksnark} require a lengthy process of initial bootstrapping.
In systems such as \textit{zcash}\footnote{\url{https://z.cash}}, this bootstrapping has been performed by a \gls{mpc}~\cite{snarkparameters}.
However, the \gls{mpc} scales poorly because of the many rounds of communication needed between parties alongside big amounts of data to ensure a fair output.
\citet{mpcsnarks} suggest avoiding the heavy communication and computation, and instead propose a simpler \gls{mpc} where users directly contribute a random number.
To avoid the last user carefully choosing their input to manipulate the bootstrapping to their benefit (a so-called \emph{last-draw attack}), an output from a randomness beacon is applied as the last input.
As such, they decrease the number of rounds in the \gls{mpc} protocol from four (plus several subprotocol rounds) to just two, decrease the amount of communication between users, and decrease the complexity of the computation.
Thus a lengthy \gls{mpc} is substituted with a quick round of user input and sealing the deal with a randomness beacon.
