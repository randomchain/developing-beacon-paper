\subsection{Sampling}

Many use cases are a form of sampling. In quality control, a few items sampled from a large group must be tested because either the test destroys the item under test or the test is expensive per item. Conflicting interests can bias the selection of items. We give an example of water samples from a river being tested for contamination. If the major industrial plants upriver can bias the random time the water samples are collected, they can potentially choose a time where the contamination is underestimated, resulting in environmental damage. On the other hand, environmental activists may tamper with the random time to show an overestimation of contamination, resulting in loss for the industrial plants upriver.

For a different example, consider random selections made by the government, such as assigning children to schools or selecting civilians for military service. Ensuring the fairness of such a selection is vital not just to the citizens but also their trust in the government.
Additionally, lotteries can be viewed as form of sampling, where a random winner is picked from a group of participants. Here the incentive for the participants to somehow bias the lottery is obvious; winning the grand prize. As such, having a verifiably random beacon that all participants can see, but not bias to their benefit, is important.
