\section{Security Analysis}%
\label{sec:security_analysis}

% Threat model --- who are we protecting against?
%   Operator (inside)
%       We cannot safeguard against an op. who shutdown the beacon
%       able to benefit from keeping output to themself
%       able to profit from selling early access to output
%       able to manipulate inputs such that output can be biased or predicted
%   Adversaries (outside, bad intentions)
%       provide input which can manipulate the output to be bias or predictable
%       deny service for other users by providing overwhelming amount of inputs
%       deny beacon operation

% Threat categorization --- use STRIDE
%   Spoofing identity
%       Not really a problem since users need not to be authorized to provide input
%   Tampering with data
%       Operator can manipulate inputs or the arrangement of them to change the output
%       Tampering with the output (both op. and adversary)
%   Repudiation
%       Operator can precompute outputs for seemingly valid (but fake) inputs
%       Operator can provide exclusive early access to outputs
%       Operator can favour some inputs
%   Information disclosure
%       Not really relevant in our case, due to total transparency
%       Might be a problem if user privacy is relevant?
%   Denial of service
%       Operator can ignore some users, thereby denying their participation
%       Operator can choose to halt operation or not publish output
%       Adversary can overwhelm beacon with inputs, either crashing it, or denying other users from providing input
%       Adversary can eclipse certain users
%       Adversary can deny service to outlets, thereby delaying or preventing output from being published
%   Elevation of privilege
%       Not really relevant, since no user has different privilege
%       Perhaps relevant if adversary can hack operator and control beacon --- then all threats by beacon operator are possible for said adversary

\subimport{}{availability.tex}
