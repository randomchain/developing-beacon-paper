\subsection{STRIDE Threat Categorization}
\rene{Cool: you are making a STRIDE analysis on the abstract architecture of the proposed system; you should also make one for the concrete (implemented) system}

We categorize the threats towards the beacon using the STRIDE framework. STRIDE consists of six categories of threats to a given system: Spoofing identity, Tampering with data, Repudiation, Information Disclosure, Denial of Service, and Elevation of privilege. We categorize each of the threats according to this system to better structure them. This categorization contains threats from both types of adversaries, both external and internal from the operator.

\paragraph{Spoofing Identity} There are no threats in this category, as there are no identities other than the operator.

\rene{... and no-one can spoof the identity of the operator?}

\paragraph{Tampering with Data}
\begin{itemize}
    \item Operator can manipulate inputs or their arrangement to change the output.
    \item Adversaries can submit inputs in attempts to bias the output.
\end{itemize}
\rene{... users can tamper with other users input-data
... users can tamper with other users output-data}

\paragraph{Repudiation}
\begin{itemize}
    \item Operator can precompute outputs for seemingly valid (but fake) inputs \rene{This is not a problem if users check that their own input has been used?}.
    \item Operator can provide early access to inputs.
    \item Operator can favor certain inputs.
\end{itemize}
\paragraph{Information Disclosure} This is not a threat if the beacon is designed to be transparent.

\rene{... I would argue that the "operator selling early access" also falls in this category}

\paragraph{Denial of Service}
\begin{itemize}
    \item Operator can ignore certain users or their inputs.
    \item Operator can seize operations.
    \item Adversaries can overwhelm the beacon with inputs.
    \item Adversaries can eclipse users.
    \item Adversary can deny service out outlets to delay or prevent publication of the input.
\end{itemize}

\paragraph{Elevation of privilege}
\begin{itemize}
    \item Adversaries can corrupt the operator.
    \item Adversaries can hack their way into the operators position.
\end{itemize}

These make up an ordered list of threats to a public randomness beacon. These will serve as guidelines for a secure design of a beacon that prevents or mitigates as many of these as possible.
%   Spoofing identity
%       Not really a problem since users need not to be authorized to provide input
%   Tampering with data
%       Operator can manipulate inputs or the arrangement of them to change the output
%       Tampering with the output (both op. and adversary)
%   Repudiation
%       Operator can precompute outputs for seemingly valid (but fake) inputs
%       Operator can provide exclusive early access to outputs
%       Operator can favour some inputs
%   Information disclosure
%       Not really relevant in our case, due to total transparency
%       Might be a problem if user privacy is relevant?
%   Denial of service
%       Operator can ignore some users, thereby denying their participation
%       Operator can choose to halt operation or not publish output
%       Adversary can overwhelm beacon with inputs, either crashing it, or denying other users from providing input
%       Adversary can eclipse certain users
%       Adversary can deny service to outlets, thereby delaying or preventing output from being published
%   Elevation of privilege
%       Not really relevant, since no user has different privilege
%       Perhaps relevant if adversary can hack operator and control beacon --- then all threats by beacon operator are possible for said adversary
