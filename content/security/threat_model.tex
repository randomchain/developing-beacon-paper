\subsection{Threat Model}
As part of analyzing the security of a randomness beacon, we must first define what type of adversaries threaten the beacon. We identify two distinct types of adversaries to a randomness beacon: insiders and outsiders. Insiders inherently consist of a single entity --- the beacon operator. It is difficult to defend against a malicious operator, as they have direct control of the beacon, which gives them access to a variety of attacks:

\begin{description}
    \item [Shutdown] A malicious beacon operator can shut the beacon down, completely denying availability.
    \item [Withholding Output] The operator can withhold outputs that are not favorable to his interests.
    \item [Selling Early Access] The operator can give access to the output earlier to some parties than others.
    \item [Input Manipulation] The operator can manipulate the input to bias the output of the beacon. He can also selectively exclude inputs from certain sources to deny them availability.

\end{description}

It should be noted that these attacks could be performed for the benefit of other users, if they manage to corrupt the beacon operator through bribes or similar means.

The other type of adversary is the outsider, which encompasses any regular user of the beacon. These are attacks that could reasonably be to an individual or smaller organization, as we do not consider organizations that could directly control internet connections, e.g.\ ISPs or intelligence agencies. We do however assume that the adversary has enough resources to perform the following attacks:

\begin{description}
    \item [Input Biasing Attack] Adversaries can attempt to provide input that bias the output to their benefit. Most commonly this will be a last-draw attack, where they attempt to provide the last input to bias the output. 
    \item [Denial of Service] Adversaries can attempt to overwhelm the beacon with inputs to prevent other users from contributing their own input, thus denying service. Another approach could be perform a standard \gls{dos} attack on the central server of the beacon to prevent operation, or performing an eclipse attack on certain users to deny them access. 
    \item [Hijacking] Attackers can attempt to seize control of the beacon, either by directly gaining access to the beacon and all accompanying operator privileges, or indirectly by corrupting the operator. If successful, this allows them to perform all of the insider attacks previously listed.  
\end{description}



% Threat model --- who are we protecting against?
%   Operator (inside)
%       We cannot safeguard against an op. who shutdown the beacon
%       able to benefit from keeping output to themself
%       able to profit from selling early access to output
%       able to manipulate inputs such that output can be biased or predicted
%   Adversaries (outside, bad intentions)
%       provide input which can manipulate the output to be bias or predictable
%       deny service for other users by providing overwhelming amount of inputs
%       deny beacon operation
