
\subsection{Lotteries}
A classic use for randomness is sampling, however most cases of sampling are performed in environments with some levels of trust.
In such cases, a beacon would not be needed, but there are some where it would.

A lottery would be a natural fit for a beacon, as it naturally has incentives to win based on the prizes. It is also fundamentally a type of sampling to find a winner, so it provides an excellent foundation of beacon application ceremonies.

We consider ceremonies for two types of lottery applications: one between a group of friends and one with a weekly lottery service. We illustrate the differences between the two, and how to structure the use of the beacon to obtain secure randomness. Finally we also consider how to use the beacon for applications that require randomness at a higher frequency.

Consider a group of three friends that hold a small lottery among themselves. To prevent cheating (however tongue-in-cheek it might be) they decide to use our beacon to provide the deciding randomness. Recall that a user can only trust an output that they have inputted to --- so all three of them must be able to find their inputs in an output to use it.

Practically, each friend announces a commit to a specific input before repeatedly inputting it to the beacon. They should then take the first output that contains all three of their inputs and use that to determine who wins.

While this approach of having everyone input works for small groups, we cannot reasonably expect every participant in large groups to repeatedly input to the beacon until a number is found.

Instead, we consider a weekly lottery that is open for anyone to purchase tickets. We cannot wait for simultaneous inputs from all customers, so we instead use the set of beacon outputs produced during the lottery, giving customers a larger time-frame to influence the outcome.

Practically, we need some way to signal the last input to use, so the lottery must announce a commit to a \textit{stop message} at the beginning of the lottery.

The lottery then collects all beacon outputs made during the duration of the lottery and concatenates them for use in the final decision. Once the lottery nears its end, the lottery repeatedly contributes a signed version of the previously mentioned stop message. Once a commit from the beacon to that input has been seen, the lottery then announces both the signed version of the message, and its hash. The beacon output containing the stop message is then the final one used for the lottery, which can now determine its winner(s).

This scheme relies on the lottery being able to end their collection of input securely, and gives four guarantees to users. First of all, they can verify the presence of the signed stop message in the output by checking for the published hash. Second, they can verify that the stop message was sent by the lottery, as they can decrypt the signed message to obtain the original stop message. Third, they can be certain that the lottery did not craft the stop message to bias the input in a last-draw attack, as it was committed to at the beginning of the lottery. Finally, they can also be sure that adversaries did not craft a last-draw attack around the stop message, as they did not know the signed message beforehand.

Users will also have a large opportunity to input to the beacon to influence the final output. This ceremony extends the first to accept inputs from a large group over an extended period of time.


One could also consider using the customers ticket purchases as input to the beacon, ensuring all customers have influenced the output.

This specific method would also be useful for applications that require very high-frequency randomness. Consider an online casino, where a large amount of people are constantly playing games. Each of these will require randomness to determine the outcome of their games, but do not have time to wait long for an input. Neither will they want to actively contribute to the randomness, as they will be focusing on their games.

This will require a very high-frequency beacon to constantly provide fresh randomness, as well as some automated method for users to input while playing.