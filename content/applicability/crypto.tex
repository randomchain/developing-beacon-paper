\subsection{Cryptography}
Some cryptographic concepts can benefit from using a randomness beacon, namely parameter generation and protocol bootstrapping.
Many cryptographic protocols and schemas require some parameters to initialize.
Choosing these can be a lengthy process~\cite{mpcsnarks}, but also requires a great deal of trust as they can contain backdoors if crafted meticulously~\cite{nist2014backdoor}.

Alternatively the parameters could be pseudo-randomly generated by a generator that only generated good parameters.
The generator could then be seeded by a randomness beacon, as described by \citet{baigneres2015trap}.
This could be accomplished much like the large lottery ceremony by announcing a collection period and stop message.
Using all inputs collected within that period as the seed would then make a wide variety of interested parties able to input to the parameters, giving them some measure of trust in the protocol.

\subsubsection{Bootstrapping Protocols}%
\label{ssub:bootstrapping_protocols}
Another use case is bootstrapping for \gls{zksnark} systems.
Such systems require a \emph{common reference string} that must be generated as part of the bootstrapping process.
Generating this string can be an extremely complicated process as the trust of the entire system rests on the string.
Should any party possess the complete data from which the string was generated they can effectively fake proofs of anything, undermining the system.

The process requires users to trust at least one of the participants of the bootstrapping ceremony.
Because of the complexity it is difficult to scale to more than a handful of participants.
Using a randomness beacon allows the process to scale far beyond the norm, as demonstrated by~\citet{mpcsnarks}.
They present a \acrshort{mpc} protocol for \acrshort{zksnark} bootstrapping.
The protocol operates across two rounds and each includes an input from a randomness beacon. %Checked and correct

Practically, each round could be organized around a number of beacon outputs containing specific \textit{round number} messages.
These messages are signed and committed similarly to previously mentioned stop messages in \cref{ssub:large_lotteries}.
This extends the period of the rounds beyond that of our beacon's output intervals, and ensures that the completion of the bootstrapping protocol is not disrupted by a missing beacon output.

