\section{Code Structure of our Randomness Beacon}%
\label{app:beacon_code}

Code for the beacon can be found at our GitHub repository\footnote{\url{https://github.com/randomchain/randbeacon}}, and latest commit as of writing this is \code{284d3241c0eedd05f80b5014fa97dfbc3a9bfdbf}.

\bigskip\noindent
The concrete implementation of our \gls{poc} randomness beacon, is structured as a python package named \textit{randbeacon} --- see \cref{fig:randbeacon_files} for a visual representation.
In this package, each group of components, i.e.\ input collecting, input processing, computation, and publishing, is its own module.
Even though no single component relies on any other, with the exception of the \texttt{utils} module for code reuse, this module structure makes it far more convenient to deploy a beacon instance during the development phase.

To orchestrate said deployment, we use the terminal multiplexer \textit{tmux}\footnote{\url{https://github.com/tmux/tmux}} and a python program called \textit{tmuxp}\footnote{\url{https://tmuxp.git-pull.com/}}, which allows us to easily specify a beacon configuration in a yaml file.
This way of deploying through \textit{tmux} is intended as a means of debugging and demoing the beacon.

Dependencies are managed with \textit{pipenv}\footnote{\url{https://docs.pipenv.org/}} and a Pipfile, thus encouraging the use of virtual python environments.
Besides the \textit{randbeacon} package, the beacon relies on a proxy written in C, which can be found in the \texttt{proxy} directory.
Outside the repository for the randomness beacon, we use our implementation of the \textit{sloth} delay function\footnote{\url{https://github.com/randomchain/pysloth/}} and our own fork of \textit{pymerkletools}\footnote{\url{https://github.com/randomchain/pymerkletools/}}.
External dependencies are fetched from the official Python package repository\footnote{\url{https://pypi.org/}}.


{%
    \tikzstyle{every node}=[font=\ttfamily, draw=black,thick,anchor=west]
\tikzstyle{selected}=[draw=GoogleIndigo,fill=GoogleBlue]
\footnotesize
\begin{figure}
\centering
\begin{tikzpicture}[%
  grow via three points={one child at (0.3,-0.7) and
  two children at (0.3,-0.7) and (0.3,-1.4)},
  edge from parent path={(\tikzparentnode.south) |- (\tikzchildnode.west)}]
  \node {randbeacon}
    child { node {\_\_init\_\_.py}}
    child { node {computation}
        child { node {\_\_init\_\_.py} }
        child { node [selected] {delay\_sloth.py}}
    }
    child [missing] {} child [missing] {}
    child { node {input\_collection}
      child { node {\_\_init\_\_.py}}
      child { node [selected] {urandom.py}}
      child { node [selected] {simple\_tcp.py}}
      child { node [selected] {simple\_http.py}}
      child { node {telegram\_bot.py}}
    }
    child [missing] {} child [missing] {} child [missing] {} child [missing] {} child [missing] {}
    child { node {input\_processing}
      child { node {\_\_init\_\_.py}}
      child { node {concat\_sha512.py}}
      child { node [selected] {merkle.py}}
    }
    child [missing] {} child [missing] {} child [missing] {}
    child { node {publishing}
      child { node {\_\_init\_\_.py}}
      child { node [selected] {json\_dump.py}}
      child { node {twitter\_bot.py}}
    }
    child [missing] {} child [missing] {} child [missing] {}
    child { node {utils.py}};
\end{tikzpicture}
\caption{Structure of core beacon components. Placed under one python package, with modules for each group. The blue files indicate those components used in the current \gls{poc} deployment demo.}\label{fig:randbeacon_files}
\end{figure}
}
