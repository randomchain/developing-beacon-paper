\section{Details of Drand Phases}%
\label{app:details_of_drand_phases}

\paragraph{Setup Phase}
During the setup phase each node to be included in the beacon, generates a public/private key-pair, to be used long-term.
A file called the \textit{group file} is then created, consisting of all participants' public keys, and configuration metadata regarding the beacon operation.

The \textit{group file} is then distributed amongst the nodes, whom then participate in a \gls{dkg} protocol.
This protocol creates a collective public key, and a sharded private key, with each node in possession of a unique shard --- which in turn is used for the internal cryptographic operations of Drand.

\paragraph{Randomness Generation Phase}
Any node may function as a leader and initiate the randomness generation phase by broadcasting a message consisting of a time stamp to all nodes.
This time stamp message is then signed by all participating nodes with a threshold version of the \gls{bls} signature scheme.
The threshold version allows any node to construct the full signature, which is the random output, given that enough nodes has provided their shard signature.

Output values can be verified by using the public key generated in the distributed key generation.
