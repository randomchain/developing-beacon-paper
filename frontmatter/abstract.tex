Randomness beacons are services that emit a practically random number at a known, fixed interval. While we think no \enquote{killer use case} exists for these beacons, we see many potential smaller use cases, which combined merits the idea of an implementation of a randomness beacon.

A recent trend in these beacons is how to make them so transparent or secure that no party can influence the output without being detected. While some research and literature exist, there still is no implementation ready for wide public usage. We seek to design and implement a randomness beacon capable of supporting a broad range of use cases.% As such, we carefully assess and choose appropriate architectures, algorithms, and tools, that we combine into a functioning and highly scalable beacon.

Our randomness beacon is based on a service-based architecture. It sources its input from a varying number of input sources and similarly emits its output (the random number) to a varying number of publishers. The transformation from input to output is done using modular square roots, a technique which produces a witness that provides a shortcut for subsequent verification. All in all, our beacon provides security properties that allow users to trust the output, provided they themselves can verify that their own input has been part of the output.

\mtjnote{Security analysis}