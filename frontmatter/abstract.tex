Randomness beacons are services that emit a random number at a regular interval.
A recent trend in these beacons is making them transparent or secure such that no party can covertly influence the output without detection.
Existing literature on the subject lacks a bridge from their respective theoretical solution to a practical implementation suitable for deployment and usage.
Much of the existing literature also lacks a structured security analysis.

We close these gaps by designing and implementing a randomness beacon supporting a broad range of use cases.
The randomness beacon is designed to be practical in the real world while prioritizing the security and integrity of the output.

Our randomness beacon is based on a service oriented architecture and supports a multitude of input and output channels.
The transformation from input to output utilizes a \gls{cco} workflow combined with a delay function.
Together, these provide good security guarantees even under the assumption that everybody else is colluding against you.
Our beacon greatly minimizes the amount of trust needed in such a way that each user can decide how much they want to trust.
As such, even fastidious users can be serviced by our randomness beacon.

Lastly, we explore a variety of applications for our randomness beacon as a cryptographic primitive, and discuss how to use it in a secure way.

\glsresetall

